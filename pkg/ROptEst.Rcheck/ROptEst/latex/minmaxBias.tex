\HeaderA{minmaxBias}{Generic Function for the Computation of Bias-Optimally Robust ICs}{minmaxBias}
\aliasA{minmaxBias,RealRandVariable,ContNeighborhood,BiasType-method}{minmaxBias}{minmaxBias,RealRandVariable,ContNeighborhood,BiasType.Rdash.method}
\aliasA{minmaxBias,UnivariateDistribution,ContNeighborhood,asymmetricBias-method}{minmaxBias}{minmaxBias,UnivariateDistribution,ContNeighborhood,asymmetricBias.Rdash.method}
\aliasA{minmaxBias,UnivariateDistribution,ContNeighborhood,BiasType-method}{minmaxBias}{minmaxBias,UnivariateDistribution,ContNeighborhood,BiasType.Rdash.method}
\aliasA{minmaxBias,UnivariateDistribution,TotalVarNeighborhood,BiasType-method}{minmaxBias}{minmaxBias,UnivariateDistribution,TotalVarNeighborhood,BiasType.Rdash.method}
\aliasA{minmaxBias-methods}{minmaxBias}{minmaxBias.Rdash.methods}
\begin{Description}\relax
Generic function for the computation of bias-optimally robust ICs 
in case of infinitesimal robust models. This function is 
rarely called directly.
\end{Description}
\begin{Usage}
\begin{verbatim}
minmaxBias(L2deriv, neighbor, biastype, ...)

## S4 method for signature 'UnivariateDistribution,
##   ContNeighborhood, BiasType':
minmaxBias(L2deriv, neighbor, biastype = symmetricBias(), symm, Finfo, trafo, 
             upper, maxiter, tol, warn)

## S4 method for signature 'UnivariateDistribution,
##   ContNeighborhood, asymmetricBias':
minmaxBias(L2deriv, neighbor, biastype = asymmetricBias(), symm, Finfo, trafo, 
             upper, maxiter, tol, warn)

## S4 method for signature 'UnivariateDistribution,
##   TotalVarNeighborhood, BiasType':
minmaxBias(L2deriv, neighbor, biastype = symmetricBias(), symm, Finfo, trafo, 
             upper, maxiter, tol, warn)

## S4 method for signature 'RealRandVariable,
##   ContNeighborhood, BiasType':
minmaxBias(L2deriv, neighbor, biastype = symmetricBias(), Distr, DistrSymm, L2derivSymm, 
             L2derivDistrSymm, Finfo, z.start, A.start, trafo, upper, maxiter, tol, warn)

\end{verbatim}
\end{Usage}
\begin{Arguments}
\begin{ldescription}
\item[\code{L2deriv}] L2-derivative of some L2-differentiable family 
of probability measures. 
\item[\code{neighbor}] object of class \code{"Neighborhood"}. 
\item[\code{biastype}] object of class \code{"BiasType"}. 
\item[\code{...}] additional parameters. 
\item[\code{Distr}] object of class \code{"Distribution"}. 
\item[\code{symm}] logical: indicating symmetry of \code{L2deriv}. 
\item[\code{DistrSymm}] object of class \code{"DistributionSymmetry"}. 
\item[\code{L2derivSymm}] object of class \code{"FunSymmList"}. 
\item[\code{L2derivDistrSymm}] object of class \code{"DistrSymmList"}. 
\item[\code{Finfo}] Fisher information matrix. 
\item[\code{z.start}] initial value for the centering constant. 
\item[\code{A.start}] initial value for the standardizing matrix. 
\item[\code{trafo}] matrix: transformation of the parameter. 
\item[\code{upper}] upper bound for the optimal clipping bound. 
\item[\code{maxiter}] the maximum number of iterations. 
\item[\code{tol}] the desired accuracy (convergence tolerance).
\item[\code{warn}] logical: print warnings. 
\end{ldescription}
\end{Arguments}
\begin{Value}
The bias-optimally robust IC is computed.
\end{Value}
\begin{Section}{Methods}
\describe{
\item[L2deriv = "UnivariateDistribution", neighbor = "ContNeighborhood", 
biastype = "BiasType"] computes the bias optimal influence curve for symmetric bias for L2 differentiable 
parametric families with unknown one-dimensional parameter. 

\item[L2deriv = "UnivariateDistribution", neighbor = "ContNeighborhood", 
biastype = "asymmetricBias"] computes the bias optimal influence curve for asymmetric bias for L2 differentiable 
parametric families with unknown one-dimensional parameter. 

\item[L2deriv = "UnivariateDistribution", neighbor = "TotalVarNeighborhood", 
biastype = "BiasType"] computes the bias optimal influence curve for symmetric bias for L2 differentiable 
parametric families with unknown one-dimensional parameter. 

\item[L2deriv = "RealRandVariable", neighbor = "ContNeighborhood", 
biastype = "BiasType"] computes the bias optimal influence curve for symmetric bias for L2 differentiable 
parametric families with unknown \eqn{k}{}-dimensional parameter 
(\eqn{k > 1}{}) where the underlying distribution is univariate. 

}
\end{Section}
\begin{Author}\relax
Matthias Kohl \email{Matthias.Kohl@stamats.de},
Peter Ruckdeschel \email{Peter.Ruckdeschel@uni-bayreuth.de}
\end{Author}
\begin{References}\relax
Rieder, H. (1980) Estimates derived from robust tests. Ann. Stats. \bold{8}: 106--115.

Rieder, H. (1994) \emph{Robust Asymptotic Statistics}. New York: Springer.

Ruckdeschel, P. (2005) Optimally One-Sided Bounded Influence Curves.
Mathematical Methods in Statistics \emph{14}(1), 105-131.

Kohl, M. (2005) \emph{Numerical Contributions to the Asymptotic Theory of Robustness}. 
Bayreuth: Dissertation.
\end{References}
\begin{SeeAlso}\relax
\code{\LinkA{InfRobModel-class}{InfRobModel.Rdash.class}}
\end{SeeAlso}

