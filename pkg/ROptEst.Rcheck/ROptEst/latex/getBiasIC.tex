\HeaderA{getBiasIC}{Generic function for the computation of the asymptotic bias for an IC}{getBiasIC}
\aliasA{getBiasIC,IC,ContNeighborhood,L2ParamFamily,BiasType-method}{getBiasIC}{getBiasIC,IC,ContNeighborhood,L2ParamFamily,BiasType.Rdash.method}
\aliasA{getBiasIC,IC,ContNeighborhood,missing,BiasType-method}{getBiasIC}{getBiasIC,IC,ContNeighborhood,missing,BiasType.Rdash.method}
\aliasA{getBiasIC,IC,TotalVarNeighborhood,L2ParamFamily,BiasType-method}{getBiasIC}{getBiasIC,IC,TotalVarNeighborhood,L2ParamFamily,BiasType.Rdash.method}
\aliasA{getBiasIC,IC,TotalVarNeighborhood,missing,BiasType-method}{getBiasIC}{getBiasIC,IC,TotalVarNeighborhood,missing,BiasType.Rdash.method}
\aliasA{getBiasIC-methods}{getBiasIC}{getBiasIC.Rdash.methods}
\begin{Description}\relax
Generic function for the computation of the asymptotic bias for an IC.
\end{Description}
\begin{Usage}
\begin{verbatim}
getBiasIC(IC, neighbor, L2Fam, biastype, ...)

## S4 method for signature 'IC, ContNeighborhood, missing,
##   BiasType':
getBiasIC(IC, neighbor, 
          biastype = symmetricBias(), tol = .Machine$double.eps^0.25)

## S4 method for signature 'IC, ContNeighborhood,
##   L2ParamFamily, BiasType':
getBiasIC(IC, neighbor, 
          L2Fam, biastype = symmetricBias(), tol = .Machine$double.eps^0.25)

## S4 method for signature 'IC, TotalVarNeighborhood,
##   missing, BiasType':
getBiasIC(IC, neighbor, 
          biastype = symmetricBias(), tol = .Machine$double.eps^0.25)

## S4 method for signature 'IC, TotalVarNeighborhood,
##   L2ParamFamily, BiasType':
getBiasIC(IC, neighbor, 
          L2Fam, biastype = symmetricBias(), tol = .Machine$double.eps^0.25)
\end{verbatim}
\end{Usage}
\begin{Arguments}
\begin{ldescription}
\item[\code{IC}] object of class \code{"InfluenceCurve"} 
\item[\code{neighbor}] object of class \code{"Neighborhood"}. 
\item[\code{L2Fam}] object of class \code{"L2ParamFamily"}. 
\item[\code{biastype}] object of class \code{"BiasType"}. 
\item[\code{...}] additional parameters 
\item[\code{tol}] the desired accuracy (convergence tolerance).
\end{ldescription}
\end{Arguments}
\begin{Details}\relax
To make sure that the results are valid, it is recommended
to include an additional check of the IC properties of \code{IC} 
using \code{checkIC}.
\end{Details}
\begin{Value}
The asymptotic bias of an IC is computed.
\end{Value}
\begin{Section}{Methods}
\describe{

\item[IC = "IC", risk = "asBias", neighbor = "ContNeighborhood", L2Fam = "missing", 
biastype = "BiasType"] asymptotic bias of \code{IC} in case of convex contamination neighborhoods 
and symmetric bias. 

\item[IC = "IC", risk = "asBias", neighbor = "ContNeighborhood", L2Fam = "missing", 
biastype = "BiasType"] asymptotic bias of \code{IC} under \code{L2Fam} 
in case of convex contamination neighborhoods and symmetric bias. 

\item[IC = "IC", risk = "asBias", neighbor = "TotalVarNeighborhood", L2Fam = "missing", 
biastype = "BiasType"] asymptotic bias of \code{IC} in case of total variation neighborhoods 
and symmetric bias. 

\item[IC = "IC", risk = "asBias", neighbor = "TotalVarNeighborhood", L2Fam = "missing", 
biastype = "BiasType"] asymptotic bias of \code{IC} under \code{L2Fam} 
in case of total variation neighborhoods and symmetric bias. 
}
\end{Section}
\begin{Note}\relax
This generic function is still under construction.
\end{Note}
\begin{Author}\relax
Matthias Kohl \email{Matthias.Kohl@stamats.de},
Peter Ruckdeschel \email{Peter.Ruckdeschel@uni-bayreuth.de}
\end{Author}
\begin{References}\relax
Rieder, H. (1994) \emph{Robust Asymptotic Statistics}. New York: Springer.

Kohl, M. (2005) \emph{Numerical Contributions to the Asymptotic Theory of Robustness}. 
Bayreuth: Dissertation.

Ruckdeschel, P. (2005) Optimally One-Sided Bounded Influence Curves.
Mathematical Methods in Statistics \emph{14}(1), 105-131.
\end{References}
\begin{SeeAlso}\relax
\code{\LinkA{getRiskIC-methods}{getRiskIC.Rdash.methods}}, \code{\LinkA{InfRobModel-class}{InfRobModel.Rdash.class}}
\end{SeeAlso}

