\HeaderA{getIneffDiff}{Generic Function for the Computation of Inefficiency Differences}{getIneffDiff}
\aliasA{getIneffDiff,numeric,L2ParamFamily,UncondNeighborhood,asMSE,BiasType-method}{getIneffDiff}{getIneffDiff,numeric,L2ParamFamily,UncondNeighborhood,asMSE,BiasType.Rdash.method}
\aliasA{getIneffDiff-methods}{getIneffDiff}{getIneffDiff.Rdash.methods}
\begin{Description}\relax
Generic function for the computation of inefficiency differencies.
This function is rarely called directly. It is used to compute
the radius minimax IC and the least favorable radius.
\end{Description}
\begin{Usage}
\begin{verbatim}
getIneffDiff(radius, L2Fam, neighbor, risk, biastype, ...)

## S4 method for signature 'numeric, L2ParamFamily,
##   UncondNeighborhood, asMSE, BiasType':
getIneffDiff(
          radius, L2Fam, neighbor, risk, biastype = symmetricBias(), loRad, upRad, 
            loRisk, upRisk, z.start = NULL, A.start = NULL, upper.b, MaxIter, eps, warn)
\end{verbatim}
\end{Usage}
\begin{Arguments}
\begin{ldescription}
\item[\code{radius}] neighborhood radius. 
\item[\code{L2Fam}] L2-differentiable family of probability measures. 
\item[\code{neighbor}] object of class \code{"Neighborhood"}. 
\item[\code{risk}] object of class \code{"RiskType"}. 
\item[\code{biastype}] object of class \code{"BiasType"}. 
\item[\code{...}] additional parameters 
\item[\code{loRad}] the lower end point of the interval to be searched. 
\item[\code{upRad}] the upper end point of the interval to be searched. 
\item[\code{loRisk}] the risk at the lower end point of the interval. 
\item[\code{upRisk}] the risk at the upper end point of the interval. 
\item[\code{z.start}] initial value for the centering constant. 
\item[\code{A.start}] initial value for the standardizing matrix. 
\item[\code{upper.b}] upper bound for the optimal clipping bound. 
\item[\code{MaxIter}] the maximum number of iterations 
\item[\code{eps}] the desired accuracy (convergence tolerance).
\item[\code{warn}] logical: print warnings. 
\end{ldescription}
\end{Arguments}
\begin{Value}
The inefficieny difference between the left and
the right margin of a given radius interval is computed.
\end{Value}
\begin{Section}{Methods}
\describe{
\item[radius = "numeric", L2Fam = "L2ParamFamily", 
neighbor = "UncondNeighborhood", risk = "asMSE", biastype = "BiasType":] computes difference of asymptotic MSE--inefficiency for
the boundaries of a given radius interval.
}
\end{Section}
\begin{Author}\relax
Matthias Kohl \email{Matthias.Kohl@stamats.de},
Peter Ruckdeschel \email{Peter.Ruckdeschel@uni-bayreuth.de}
\end{Author}
\begin{References}\relax
Rieder, H., Kohl, M. and Ruckdeschel, P. (2001) The Costs of not Knowing
the Radius. Submitted. Appeared as discussion paper Nr. 81. 
SFB 373 (Quantification and Simulation of Economic Processes),
Humboldt University, Berlin; also available under
\url{www.uni-bayreuth.de/departments/math/org/mathe7/RIEDER/pubs/RR.pdf}

Ruckdeschel, P. (2005) Optimally One-Sided Bounded Influence Curves.
Mathematical Methods in Statistics \emph{14}(1), 105-131.

Kohl, M. (2005) \emph{Numerical Contributions to the Asymptotic Theory of Robustness}. 
Bayreuth: Dissertation.
\end{References}
\begin{SeeAlso}\relax
\code{\LinkA{radiusMinimaxIC}{radiusMinimaxIC}}, \code{\LinkA{leastFavorableRadius}{leastFavorableRadius}}
\end{SeeAlso}

