\HeaderA{getFixClip}{Generic Function for the Computation of the Optimal Clipping Bound}{getFixClip}
\aliasA{getFixClip,numeric,Norm,fiUnOvShoot,ContNeighborhood-method}{getFixClip}{getFixClip,numeric,Norm,fiUnOvShoot,ContNeighborhood.Rdash.method}
\aliasA{getFixClip,numeric,Norm,fiUnOvShoot,TotalVarNeighborhood-method}{getFixClip}{getFixClip,numeric,Norm,fiUnOvShoot,TotalVarNeighborhood.Rdash.method}
\aliasA{getFixClip-methods}{getFixClip}{getFixClip.Rdash.methods}
\begin{Description}\relax
Generic function for the computation of the optimal clipping bound
in case of robust models with fixed neighborhoods. This function is 
rarely called directly. It is used to compute optimally robust ICs.
\end{Description}
\begin{Usage}
\begin{verbatim}
getFixClip(clip, Distr, risk, neighbor,  ...)

## S4 method for signature 'numeric, Norm, fiUnOvShoot,
##   ContNeighborhood':
getFixClip(clip, Distr, risk, neighbor)

## S4 method for signature 'numeric, Norm, fiUnOvShoot,
##   TotalVarNeighborhood':
getFixClip(clip, Distr, risk, neighbor)
\end{verbatim}
\end{Usage}
\begin{Arguments}
\begin{ldescription}
\item[\code{clip}] positive real: clipping bound 
\item[\code{Distr}] object of class \code{"Distribution"}. 
\item[\code{risk}] object of class \code{"RiskType"}. 
\item[\code{neighbor}] object of class \code{"Neighborhood"}. 
\item[\code{...}] additional parameters. 
\end{ldescription}
\end{Arguments}
\begin{Value}
The optimal clipping bound is computed.
\end{Value}
\begin{Section}{Methods}
\describe{
\item[clip = "numeric", Distr = "Norm", risk = "fiUnOvShoot", neighbor = "ContNeighborhood"] optimal clipping bound for finite-sample under-/overshoot risk. 

\item[clip = "numeric", Distr = "Norm", risk = "fiUnOvShoot", neighbor = "TotalVarNeighborhood"] optimal clipping bound for finite-sample under-/overshoot risk. 
}
\end{Section}
\begin{Author}\relax
Matthias Kohl \email{Matthias.Kohl@stamats.de}
\end{Author}
\begin{References}\relax
Huber, P.J. (1968) Robust Confidence Limits. Z. Wahrscheinlichkeitstheor.
Verw. Geb. \bold{10}:269--278.

Kohl, M. (2005) \emph{Numerical Contributions to the Asymptotic Theory of Robustness}. 
Bayreuth: Dissertation.
\end{References}
\begin{SeeAlso}\relax
\code{\LinkA{ContIC-class}{ContIC.Rdash.class}}, \code{\LinkA{TotalVarIC-class}{TotalVarIC.Rdash.class}}
\end{SeeAlso}

