\HeaderA{leastFavorableRadius}{Generic Function for the Computation of Least Favorable Radii}{leastFavorableRadius}
\aliasA{leastFavorableRadius,L2ParamFamily,UncondNeighborhood,asGRisk-method}{leastFavorableRadius}{leastFavorableRadius,L2ParamFamily,UncondNeighborhood,asGRisk.Rdash.method}
\aliasA{leastFavorableRadius-methods}{leastFavorableRadius}{leastFavorableRadius.Rdash.methods}
\begin{Description}\relax
Generic function for the computation of least favorable radii.
\end{Description}
\begin{Usage}
\begin{verbatim}
leastFavorableRadius(L2Fam, neighbor, risk, ...)

## S4 method for signature 'L2ParamFamily,
##   UncondNeighborhood, asGRisk':
leastFavorableRadius(
          L2Fam, neighbor, risk, biastype = symmetricBias(), rho, upRad = 1, 
            z.start = NULL, A.start = NULL, upper = 100, maxiter = 100, 
            tol = .Machine$double.eps^0.4, warn = FALSE)
\end{verbatim}
\end{Usage}
\begin{Arguments}
\begin{ldescription}
\item[\code{L2Fam}] L2-differentiable family of probability measures. 
\item[\code{neighbor}] object of class \code{"Neighborhood"}. 
\item[\code{risk}] object of class \code{"RiskType"}. 
\item[\code{...}] additional parameters 
\item[\code{biastype}] object of class \code{"BiasType"}. 
\item[\code{upRad}] the upper end point of the radius interval to be searched. 
\item[\code{rho}] The considered radius interval is: \eqn{[r \rho, r/\rho]}{[r*rho, r/rho]}
with \eqn{\rho\in(0,1)}{0 < rho < 1}. 
\item[\code{z.start}] initial value for the centering constant. 
\item[\code{A.start}] initial value for the standardizing matrix. 
\item[\code{upper}] upper bound for the optimal clipping bound. 
\item[\code{maxiter}] the maximum number of iterations 
\item[\code{tol}] the desired accuracy (convergence tolerance).
\item[\code{warn}] logical: print warnings. 
\end{ldescription}
\end{Arguments}
\begin{Value}
The least favorable radius and the corresponding inefficiency 
are computed.
\end{Value}
\begin{Section}{Methods}
\describe{
\item[L2Fam = "L2ParamFamily", neighbor = "UncondNeighborhood", 
risk = "asGRisk"] computation of the least favorable radius. 
}
\end{Section}
\begin{Author}\relax
Matthias Kohl \email{Matthias.Kohl@stamats.de},
Peter Ruckdeschel \email{Peter.Ruckdeschel@uni-bayreuth.de}
\end{Author}
\begin{References}\relax
Rieder, H., Kohl, M. and Ruckdeschel, P. (2008) The Costs of not Knowing
the Radius. Statistical Methods and Applications \emph{17}(1) 13-40.

Rieder, H., Kohl, M. and Ruckdeschel, P. (2001) The Costs of not Knowing
the Radius. Submitted. Appeared as discussion paper Nr. 81. 
SFB 373 (Quantification and Simulation of Economic Processes),
Humboldt University, Berlin; also available under
\url{www.uni-bayreuth.de/departments/math/org/mathe7/RIEDER/pubs/RR.pdf}

Ruckdeschel, P. (2005) Optimally One-Sided Bounded Influence Curves.
Mathematical Methods in Statistics \emph{14}(1), 105-131.

Kohl, M. (2005) \emph{Numerical Contributions to the Asymptotic Theory of Robustness}. 
Bayreuth: Dissertation.
\end{References}
\begin{SeeAlso}\relax
\code{\LinkA{radiusMinimaxIC}{radiusMinimaxIC}}
\end{SeeAlso}
\begin{Examples}
\begin{ExampleCode}
N <- NormLocationFamily(mean=0, sd=1) 
leastFavorableRadius(L2Fam=N, neighbor=ContNeighborhood(),
                     risk=asMSE(), rho=0.5)
\end{ExampleCode}
\end{Examples}

