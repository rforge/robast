\HeaderA{trFiCov-class}{Trace of finite-sample covariance}{trFiCov.Rdash.class}
\keyword{classes}{trFiCov-class}
\begin{Description}\relax
Class of trace of finite-sample covariance.
\end{Description}
\begin{Section}{Objects from the Class}
Objects can be created by calls of the form \code{new("trFiCov", ...)}.
More frequently they are created via the generating function 
\code{trFiCov}.
\end{Section}
\begin{Section}{Slots}
\describe{
\item[\code{type}:] Object of class \code{"character"}:
\dQuote{trace of finite-sample covariance}. 
}
\end{Section}
\begin{Section}{Extends}
Class \code{"fiRisk"}, directly.\\
Class \code{"RiskType"}, by class \code{"fiRisk"}.
\end{Section}
\begin{Author}\relax
Matthias Kohl \email{Matthias.Kohl@stamats.de}
\end{Author}
\begin{References}\relax
Ruckdeschel, P. and Kohl, M. (2005) How to approximate 
the finite sample risk of M-estimators.
\end{References}
\begin{SeeAlso}\relax
\code{\LinkA{fiRisk-class}{fiRisk.Rdash.class}}, \code{\LinkA{trFiCov}{trFiCov}}
\end{SeeAlso}
\begin{Examples}
\begin{ExampleCode}
new("trFiCov")
\end{ExampleCode}
\end{Examples}

