\HeaderA{getInfCent}{Generic Function for the Computation of the Optimal Centering Constant/Lower Clipping Bound}{getInfCent}
\aliasA{getInfCent,RealRandVariable,ContNeighborhood,BiasType-method}{getInfCent}{getInfCent,RealRandVariable,ContNeighborhood,BiasType.Rdash.method}
\aliasA{getInfCent,UnivariateDistribution,ContNeighborhood,asymmetricBias-method}{getInfCent}{getInfCent,UnivariateDistribution,ContNeighborhood,asymmetricBias.Rdash.method}
\aliasA{getInfCent,UnivariateDistribution,ContNeighborhood,BiasType-method}{getInfCent}{getInfCent,UnivariateDistribution,ContNeighborhood,BiasType.Rdash.method}
\aliasA{getInfCent,UnivariateDistribution,ContNeighborhood,onesidedBias-method}{getInfCent}{getInfCent,UnivariateDistribution,ContNeighborhood,onesidedBias.Rdash.method}
\aliasA{getInfCent,UnivariateDistribution,TotalVarNeighborhood,BiasType-method}{getInfCent}{getInfCent,UnivariateDistribution,TotalVarNeighborhood,BiasType.Rdash.method}
\aliasA{getInfCent-methods}{getInfCent}{getInfCent.Rdash.methods}
\begin{Description}\relax
Generic function for the computation of the optimal centering constant
(contamination neighborhoods) respectively, of the optimal lower clipping
bound (total variation neighborhood).
This function is rarely called directly. It is used to 
compute optimally robust ICs.
\end{Description}
\begin{Usage}
\begin{verbatim}
getInfCent(L2deriv, neighbor, biastype, ...)

## S4 method for signature 'UnivariateDistribution,
##   ContNeighborhood, BiasType':
getInfCent(L2deriv, 
     neighbor, biastype = symmetricBias(), clip, cent, tol.z, symm, trafo)

## S4 method for signature 'UnivariateDistribution,
##   TotalVarNeighborhood, BiasType':
getInfCent(L2deriv, 
     neighbor, biastype = symmetricBias(), clip, cent, tol.z, symm, trafo)

## S4 method for signature 'RealRandVariable,
##   ContNeighborhood, BiasType':
getInfCent(L2deriv, 
     neighbor, biastype = symmetricBias(), z.comp, stand, cent, clip)

## S4 method for signature 'UnivariateDistribution,
##   ContNeighborhood, onesidedBias':
getInfCent(L2deriv, 
     neighbor, biastype = positiveBias(), clip, cent, tol.z, symm, trafo)

## S4 method for signature 'UnivariateDistribution,
##   ContNeighborhood, asymmetricBias':
getInfCent(L2deriv, 
     neighbor, biastype = asymmetricBias(), clip, cent, tol.z, symm, trafo)
\end{verbatim}
\end{Usage}
\begin{Arguments}
\begin{ldescription}
\item[\code{L2deriv}] L2-derivative of some L2-differentiable family 
of probability measures. 
\item[\code{neighbor}] object of class \code{"Neighborhood"}. 
\item[\code{biastype}] object of class \code{"BiasType"} 
\item[\code{...}] additional parameters. 
\item[\code{clip}] optimal clipping bound. 
\item[\code{cent}] optimal centering constant. 
\item[\code{stand}] standardizing matrix. 
\item[\code{tol.z}] the desired accuracy (convergence tolerance). 
\item[\code{symm}] logical: indicating symmetry of \code{L2deriv}. 
\item[\code{trafo}] matrix: transformation of the parameter. 
\item[\code{z.comp}] logical vector: indication which components of the 
centering constant have to be computed. 
\end{ldescription}
\end{Arguments}
\begin{Value}
The optimal centering constant is computed.
\end{Value}
\begin{Section}{Methods}
\describe{
\item[L2deriv = "UnivariateDistribution", neighbor = "ContNeighborhood", biastype = "BiasType"] computation of optimal centering constant for symmetric bias. 

\item[L2deriv = "UnivariateDistribution", neighbor = "TotalVarNeighborhood", biastype = "BiasType"] computation of optimal lower clipping bound for symmetric bias. 

\item[L2deriv = "RealRandVariable", neighbor = "ContNeighborhood", biastype = "BiasType"] computation of optimal centering constant for symmetric bias. 

\item[L2deriv = "UnivariateDistribution", neighbor = "ContNeighborhood", biastype = "onesidedBias"] computation of optimal centering constant for onesided bias. 

\item[L2deriv = "UnivariateDistribution", neighbor = "ContNeighborhood", biastype = "asymmetricBias"] computation of optimal centering constant for asymmetric bias. 
}
\end{Section}
\begin{Author}\relax
Matthias Kohl \email{Matthias.Kohl@stamats.de},
Peter Ruckdeschel \email{Peter.Ruckdeschel@uni-bayreuth.de}
\end{Author}
\begin{References}\relax
Rieder, H. (1994) \emph{Robust Asymptotic Statistics}. New York: Springer.

Ruckdeschel, P. (2005) Optimally One-Sided Bounded Influence Curves.
Mathematical Methods in Statistics \emph{14}(1), 105-131.

Kohl, M. (2005) \emph{Numerical Contributions to the Asymptotic Theory of Robustness}. 
Bayreuth: Dissertation.
\end{References}
\begin{SeeAlso}\relax
\code{\LinkA{ContIC-class}{ContIC.Rdash.class}}, \code{\LinkA{TotalVarIC-class}{TotalVarIC.Rdash.class}}
\end{SeeAlso}

