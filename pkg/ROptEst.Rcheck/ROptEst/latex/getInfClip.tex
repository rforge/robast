\HeaderA{getInfClip}{Generic Function for the Computation of the Optimal Clipping Bound}{getInfClip}
\aliasA{getInfClip,numeric,EuclRandVariable,asMSE,ContNeighborhood-method}{getInfClip}{getInfClip,numeric,EuclRandVariable,asMSE,ContNeighborhood.Rdash.method}
\aliasA{getInfClip,numeric,UnivariateDistribution,asMSE,ContNeighborhood-method}{getInfClip}{getInfClip,numeric,UnivariateDistribution,asMSE,ContNeighborhood.Rdash.method}
\aliasA{getInfClip,numeric,UnivariateDistribution,asMSE,TotalVarNeighborhood-method}{getInfClip}{getInfClip,numeric,UnivariateDistribution,asMSE,TotalVarNeighborhood.Rdash.method}
\aliasA{getInfClip,numeric,UnivariateDistribution,asSemivar,ContNeighborhood-method}{getInfClip}{getInfClip,numeric,UnivariateDistribution,asSemivar,ContNeighborhood.Rdash.method}
\aliasA{getInfClip,numeric,UnivariateDistribution,asUnOvShoot,UncondNeighborhood-method}{getInfClip}{getInfClip,numeric,UnivariateDistribution,asUnOvShoot,UncondNeighborhood.Rdash.method}
\aliasA{getInfClip-methods}{getInfClip}{getInfClip.Rdash.methods}
\begin{Description}\relax
Generic function for the computation of the optimal clipping bound
in case of infinitesimal robust models. This function is rarely called 
directly. It is used to compute optimally robust ICs.
\end{Description}
\begin{Usage}
\begin{verbatim}
getInfClip(clip, L2deriv, risk, neighbor, ...)

## S4 method for signature 'numeric,
##   UnivariateDistribution, asMSE, ContNeighborhood':
getInfClip(clip, L2deriv, risk, neighbor, biastype, cent, symm, trafo)

## S4 method for signature 'numeric,
##   UnivariateDistribution, asMSE, TotalVarNeighborhood':
getInfClip(clip, L2deriv, risk, neighbor, biastype, cent, symm, trafo)

## S4 method for signature 'numeric, EuclRandVariable,
##   asMSE, ContNeighborhood':
getInfClip(clip, L2deriv, risk, neighbor, Distr, stand, biastype, cent, trafo)

## S4 method for signature 'numeric,
##   UnivariateDistribution, asUnOvShoot,
##   UncondNeighborhood':
getInfClip(clip, L2deriv, risk, neighbor, biastype, cent, symm, trafo)

## S4 method for signature 'numeric,
##   UnivariateDistribution, asSemivar, ContNeighborhood':
getInfClip(clip, L2deriv, risk, neighbor, biastype, cent, symm, trafo)
\end{verbatim}
\end{Usage}
\begin{Arguments}
\begin{ldescription}
\item[\code{clip}] positive real: clipping bound 
\item[\code{L2deriv}] L2-derivative of some L2-differentiable family 
of probability measures. 
\item[\code{risk}] object of class \code{"RiskType"}. 
\item[\code{neighbor}] object of class \code{"Neighborhood"}. 
\item[\code{...}] additional parameters. 
\item[\code{biastype}] object of class \code{"BiasType"} 
\item[\code{cent}] optimal centering constant. 
\item[\code{stand}] standardizing matrix. 
\item[\code{Distr}] object of class \code{"Distribution"}. 
\item[\code{symm}] logical: indicating symmetry of \code{L2deriv}. 
\item[\code{trafo}] matrix: transformation of the parameter. 
\end{ldescription}
\end{Arguments}
\begin{Value}
The optimal clipping bound is computed.
\end{Value}
\begin{Section}{Methods}
\describe{
\item[clip = "numeric", L2deriv = "UnivariateDistribution", 
risk = "asMSE", neighbor = "ContNeighborhood"] optimal clipping bound for asymtotic mean square error. 


\item[clip = "numeric", L2deriv = "UnivariateDistribution", 
risk = "asMSE", neighbor = "TotalVarNeighborhood"] optimal clipping bound for asymtotic mean square error. 

\item[clip = "numeric", L2deriv = "EuclRandVariable", 
risk = "asMSE", neighbor = "ContNeighborhood"] optimal clipping bound for asymtotic mean square error. 

\item[clip = "numeric", L2deriv = "UnivariateDistribution", 
risk = "asUnOvShoot", neighbor = "UncondNeighborhood"] optimal clipping bound for asymtotic under-/overshoot risk. 

\item[clip = "numeric", L2deriv = "UnivariateDistribution", 
risk = "asSemivar", neighbor = "ContNeighborhood"] optimal clipping bound for asymtotic semivariance.
}
\end{Section}
\begin{Author}\relax
Matthias Kohl \email{Matthias.Kohl@stamats.de},
Peter Ruckdeschel \email{Peter.Ruckdeschel@uni-bayreuth.de}
\end{Author}
\begin{References}\relax
Rieder, H. (1980) Estimates derived from robust tests. Ann. Stats. \bold{8}: 106--115.

Rieder, H. (1994) \emph{Robust Asymptotic Statistics}. New York: Springer.

Ruckdeschel, P. (2005) Optimally One-Sided Bounded Influence Curves.
Mathematical Methods in Statistics \emph{14}(1), 105-131.

Kohl, M. (2005) \emph{Numerical Contributions to the Asymptotic Theory of Robustness}. 
Bayreuth: Dissertation.
\end{References}
\begin{SeeAlso}\relax
\code{\LinkA{ContIC-class}{ContIC.Rdash.class}}, \code{\LinkA{TotalVarIC-class}{TotalVarIC.Rdash.class}}
\end{SeeAlso}

