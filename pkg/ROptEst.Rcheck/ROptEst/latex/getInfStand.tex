\HeaderA{getInfStand}{Generic Function for the Computation of the Standardizing Matrix}{getInfStand}
\aliasA{getInfStand,RealRandVariable,ContNeighborhood,BiasType-method}{getInfStand}{getInfStand,RealRandVariable,ContNeighborhood,BiasType.Rdash.method}
\aliasA{getInfStand,UnivariateDistribution,ContNeighborhood,asymmetricBias-method}{getInfStand}{getInfStand,UnivariateDistribution,ContNeighborhood,asymmetricBias.Rdash.method}
\aliasA{getInfStand,UnivariateDistribution,ContNeighborhood,BiasType-method}{getInfStand}{getInfStand,UnivariateDistribution,ContNeighborhood,BiasType.Rdash.method}
\aliasA{getInfStand,UnivariateDistribution,ContNeighborhood,onesidedBias-method}{getInfStand}{getInfStand,UnivariateDistribution,ContNeighborhood,onesidedBias.Rdash.method}
\aliasA{getInfStand,UnivariateDistribution,TotalVarNeighborhood,BiasType-method}{getInfStand}{getInfStand,UnivariateDistribution,TotalVarNeighborhood,BiasType.Rdash.method}
\aliasA{getInfStand-methods}{getInfStand}{getInfStand.Rdash.methods}
\begin{Description}\relax
Generic function for the computation of the standardizing matrix which
takes care of the Fisher consistency of the corresponding IC. This function 
is rarely called directly. It is used to compute optimally robust ICs.
\end{Description}
\begin{Usage}
\begin{verbatim}
getInfStand(L2deriv, neighbor, biastype, ...)

## S4 method for signature 'UnivariateDistribution,
##   ContNeighborhood, BiasType':
getInfStand(L2deriv, 
     neighbor, biastype = symmetricBias(), clip, cent, trafo)

## S4 method for signature 'UnivariateDistribution,
##   TotalVarNeighborhood, BiasType':
getInfStand(L2deriv, 
     neighbor, biastype = symmetricBias(), clip, cent, trafo)

## S4 method for signature 'RealRandVariable,
##   ContNeighborhood, BiasType':
getInfStand(L2deriv, 
     neighbor, biastype = symmetricBias(), Distr, A.comp, stand, clip, cent, trafo)

## S4 method for signature 'UnivariateDistribution,
##   ContNeighborhood, BiasType':
getInfStand(L2deriv, 
     neighbor, biastype = positiveBias(), clip, cent, trafo)

## S4 method for signature 'UnivariateDistribution,
##   ContNeighborhood, BiasType':
getInfStand(L2deriv, 
     neighbor, biastype = asymmetricBias(), clip, cent, trafo)
\end{verbatim}
\end{Usage}
\begin{Arguments}
\begin{ldescription}
\item[\code{L2deriv}] L2-derivative of some L2-differentiable family 
of probability measures. 
\item[\code{neighbor}] object of class \code{"Neighborhood"} 
\item[\code{biastype}] object of class \code{"BiasType"} 
\item[\code{...}] additional parameters 
\item[\code{clip}] optimal clipping bound. 
\item[\code{cent}] optimal centering constant. 
\item[\code{stand}] standardizing matrix. 
\item[\code{Distr}] object of class \code{"Distribution"}. 
\item[\code{trafo}] matrix: transformation of the parameter. 
\item[\code{A.comp}] matrix: indication which components of the standardizing
matrix have to be computed. 
\end{ldescription}
\end{Arguments}
\begin{Value}
The standardizing matrix is computed.
\end{Value}
\begin{Section}{Methods}
\describe{
\item[L2deriv = "UnivariateDistribution", neighbor = "ContNeighborhood", 
biastype = "BiasType"] computes standardizing matrix for symmetric bias. 

\item[L2deriv = "UnivariateDistribution", neighbor = "TotalVarNeighborhood", 
biastype = "BiasType"] computes standardizing matrix for symmetric bias. 

\item[L2deriv = "RealRandVariable", neighbor = "ContNeighborhood", 
biastype = "BiasType"] computes standardizing matrix for symmetric bias. 

\item[L2deriv = "UnivariateDistribution", neighbor = "ContNeighborhood", 
biastype = "onesidedBias"] computes standardizing matrix for onesided bias. 

\item[L2deriv = "UnivariateDistribution", neighbor = "ContNeighborhood", 
biastype = "asymmetricBias"] computes standardizing matrix for asymmetric bias. 
}
\end{Section}
\begin{Author}\relax
Matthias Kohl \email{Matthias.Kohl@stamats.de},
Peter Ruckdeschel \email{Peter.Ruckdeschel@uni-bayreuth.de}
\end{Author}
\begin{References}\relax
Rieder, H. (1994) \emph{Robust Asymptotic Statistics}. New York: Springer.

Ruckdeschel, P. (2005) Optimally One-Sided Bounded Influence Curves.
Mathematical Methods in Statistics \emph{14}(1), 105-131.

Kohl, M. (2005) \emph{Numerical Contributions to the Asymptotic Theory of Robustness}. 
Bayreuth: Dissertation.
\end{References}
\begin{SeeAlso}\relax
\code{\LinkA{ContIC-class}{ContIC.Rdash.class}}, \code{\LinkA{TotalVarIC-class}{TotalVarIC.Rdash.class}}
\end{SeeAlso}

