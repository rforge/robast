\HeaderA{getInfGamma}{Generic Function for the Computation of the Optimal Clipping Bound}{getInfGamma}
\aliasA{getInfGamma,RealRandVariable,asMSE,ContNeighborhood,BiasType-method}{getInfGamma}{getInfGamma,RealRandVariable,asMSE,ContNeighborhood,BiasType.Rdash.method}
\aliasA{getInfGamma,UnivariateDistribution,asGRisk,TotalVarNeighborhood,BiasType-method}{getInfGamma}{getInfGamma,UnivariateDistribution,asGRisk,TotalVarNeighborhood,BiasType.Rdash.method}
\aliasA{getInfGamma,UnivariateDistribution,asMSE,ContNeighborhood,asymmetricBias-method}{getInfGamma}{getInfGamma,UnivariateDistribution,asMSE,ContNeighborhood,asymmetricBias.Rdash.method}
\aliasA{getInfGamma,UnivariateDistribution,asMSE,ContNeighborhood,BiasType-method}{getInfGamma}{getInfGamma,UnivariateDistribution,asMSE,ContNeighborhood,BiasType.Rdash.method}
\aliasA{getInfGamma,UnivariateDistribution,asMSE,ContNeighborhood,onesidedBias-method}{getInfGamma}{getInfGamma,UnivariateDistribution,asMSE,ContNeighborhood,onesidedBias.Rdash.method}
\aliasA{getInfGamma,UnivariateDistribution,asUnOvShoot,ContNeighborhood,BiasType-method}{getInfGamma}{getInfGamma,UnivariateDistribution,asUnOvShoot,ContNeighborhood,BiasType.Rdash.method}
\aliasA{getInfGamma-methods}{getInfGamma}{getInfGamma.Rdash.methods}
\begin{Description}\relax
Generic function for the computation of the optimal clipping bound.
This function is rarely called directly. It is called by \code{getInfClip} 
to compute optimally robust ICs.
\end{Description}
\begin{Usage}
\begin{verbatim}
getInfGamma(L2deriv, risk, neighbor, biastype, ...)

## S4 method for signature 'UnivariateDistribution, asMSE,
##   ContNeighborhood, BiasType':
getInfGamma(L2deriv, 
     risk, neighbor, biastype = symmetricBias(), cent, clip)

## S4 method for signature 'UnivariateDistribution,
##   asGRisk, TotalVarNeighborhood, BiasType':
getInfGamma(L2deriv, 
     risk, neighbor, biastype = symmetricBias(), cent, clip)

## S4 method for signature 'RealRandVariable, asMSE,
##   ContNeighborhood, BiasType':
getInfGamma(L2deriv, 
     risk, neighbor, biastype = symmetricBias(), Distr, stand, cent, clip)

## S4 method for signature 'UnivariateDistribution,
##   asUnOvShoot, ContNeighborhood, BiasType':
getInfGamma(L2deriv, 
     risk, neighbor, biastype = symmetricBias(), cent, clip)

## S4 method for signature 'UnivariateDistribution, asMSE,
##   ContNeighborhood, onesidedBias':
getInfGamma(L2deriv, 
     risk, neighbor, biastype = positiveBias(), cent, clip)

## S4 method for signature 'UnivariateDistribution, asMSE,
##   ContNeighborhood, asymmetricBias':
getInfGamma(L2deriv, 
    risk, neighbor, biastype  = asymmetricBias(), cent, clip)
\end{verbatim}
\end{Usage}
\begin{Arguments}
\begin{ldescription}
\item[\code{L2deriv}] L2-derivative of some L2-differentiable family 
of probability measures. 
\item[\code{risk}] object of class \code{"RiskType"}. 
\item[\code{neighbor}] object of class \code{"Neighborhood"}. 
\item[\code{biastype}] object of class \code{"BiasType"} 
\item[\code{...}] additional parameters 
\item[\code{cent}] optimal centering constant. 
\item[\code{clip}] optimal clipping bound. 
\item[\code{stand}] standardizing matrix. 
\item[\code{Distr}] object of class \code{"Distribution"}. 
\end{ldescription}
\end{Arguments}
\begin{Details}\relax
The function is used in case of asymptotic G-risks; confer
Ruckdeschel and Rieder (2004).
\end{Details}
\begin{Section}{Methods}
\describe{
\item[L2deriv = "UnivariateDistribution", risk = "asMSE", 
neighbor = "ContNeighborhood", 
biastype = "BiasType"] used by \code{getInfClip} for symmetric bias. 

\item[L2deriv = "UnivariateDistribution", risk = "asGRisk", 
neighbor = "TotalVarNeighborhood", 
biastype = "BiasType"] used by \code{getInfClip} for symmetric bias. 

\item[L2deriv = "RealRandVariable", risk = "asMSE", 
neighbor = "ContNeighborhood", 
biastype = "BiasType"] used by \code{getInfClip} for symmetric bias. 

\item[L2deriv = "UnivariateDistribution", risk = "asUnOvShoot", 
neighbor = "ContNeighborhood", 
biastype = "BiasType"] used by \code{getInfClip} for symmetric bias. 

\item[L2deriv = "UnivariateDistribution", risk = "asMSE", 
neighbor = "ContNeighborhood", 
biastype = "onesidedBias"] used by \code{getInfClip} for onesided bias. 

\item[L2deriv = "UnivariateDistribution", risk = "asMSE", 
neighbor = "ContNeighborhood", 
biastype = "asymmetricBias"] used by \code{getInfClip} for asymmetric bias. 
}
\end{Section}
\begin{Author}\relax
Matthias Kohl \email{Matthias.Kohl@stamats.de},
Peter Ruckdeschel \email{Peter.Ruckdeschel@uni-bayreuth.de}
\end{Author}
\begin{References}\relax
Rieder, H. (1980) Estimates derived from robust tests. Ann. Stats. \bold{8}: 106--115.

Rieder, H. (1994) \emph{Robust Asymptotic Statistics}. New York: Springer.

Ruckdeschel, P. and Rieder, H. (2004) Optimal Influence Curves for
General Loss Functions. Statistics \& Decisions \emph{22}, 201-223.

Ruckdeschel, P. (2005) Optimally One-Sided Bounded Influence Curves.
Mathematical Methods in Statistics \emph{14}(1), 105-131.

Kohl, M. (2005) \emph{Numerical Contributions to the Asymptotic Theory of Robustness}. 
Bayreuth: Dissertation.
\end{References}
\begin{SeeAlso}\relax
\code{\LinkA{asGRisk-class}{asGRisk.Rdash.class}}, \code{\LinkA{asMSE-class}{asMSE.Rdash.class}},
\code{\LinkA{asUnOvShoot-class}{asUnOvShoot.Rdash.class}}, \code{\LinkA{ContIC-class}{ContIC.Rdash.class}}, 
\code{\LinkA{TotalVarIC-class}{TotalVarIC.Rdash.class}}
\end{SeeAlso}

