\HeaderA{optIC}{Generic function for the computation of optimally robust ICs}{optIC}
\aliasA{optIC,FixRobModel,fiUnOvShoot-method}{optIC}{optIC,FixRobModel,fiUnOvShoot.Rdash.method}
\aliasA{optIC,InfRobModel,asRisk-method}{optIC}{optIC,InfRobModel,asRisk.Rdash.method}
\aliasA{optIC,InfRobModel,asUnOvShoot-method}{optIC}{optIC,InfRobModel,asUnOvShoot.Rdash.method}
\aliasA{optIC,L2ParamFamily,asCov-method}{optIC}{optIC,L2ParamFamily,asCov.Rdash.method}
\aliasA{optIC-methods}{optIC}{optIC.Rdash.methods}
\begin{Description}\relax
Generic function for the computation of optimally robust ICs.
\end{Description}
\begin{Usage}
\begin{verbatim}
optIC(model, risk, ...)

## S4 method for signature 'L2ParamFamily, asCov':
optIC(model, risk)

## S4 method for signature 'InfRobModel, asRisk':
optIC(model, risk, biastype = symmetricBias(), 
                 z.start = NULL, A.start = NULL, upper = 1e4, 
             maxiter = 50, tol = .Machine$double.eps^0.4, warn = TRUE)

## S4 method for signature 'InfRobModel, asUnOvShoot':
optIC(model, risk, biastype = symmetricBias(), 
                 upper = 1e4, maxiter = 50, 
                 tol = .Machine$double.eps^0.4, warn = TRUE)

## S4 method for signature 'FixRobModel, fiUnOvShoot':
optIC(model, risk, sampleSize, upper = 1e4, maxiter = 50, 
             tol = .Machine$double.eps^0.4, warn = TRUE, Algo = "A", cont = "left")
\end{verbatim}
\end{Usage}
\begin{Arguments}
\begin{ldescription}
\item[\code{model}] probability model. 
\item[\code{risk}] object of class \code{"RiskType"}. 
\item[\code{...}] additional parameters. 
\item[\code{biastype}] object of class \code{"BiasType"} 
\item[\code{z.start}] initial value for the centering constant. 
\item[\code{A.start}] initial value for the standardizing matrix. 
\item[\code{upper}] upper bound for the optimal clipping bound. 
\item[\code{maxiter}] the maximum number of iterations. 
\item[\code{tol}] the desired accuracy (convergence tolerance).
\item[\code{warn}] logical: print warnings. 
\item[\code{sampleSize}] integer: sample size. 
\item[\code{Algo}] "A" or "B". 
\item[\code{cont}] "left" or "right". 
\end{ldescription}
\end{Arguments}
\begin{Details}\relax
In case of the finite-sample risk \code{"fiUnOvShoot"} one can choose
between two algorithms for the computation of this risk where the least favorable
contamination is assumed to be left or right of some bound. For more details
we refer to Section 11.3 of Kohl (2005).
\end{Details}
\begin{Value}
Some optimally robust IC is computed.
\end{Value}
\begin{Section}{Methods}
\describe{
\item[model = "L2ParamFamily", risk = "asCov"] computes
classical optimal influence curve for L2 differentiable 
parametric families.

\item[model = "InfRobModel", risk = "asRisk"] computes optimally robust influence curve for 
robust models with infinitesimal neighborhoods and
various asymptotic risks. 

\item[model = "InfRobModel", risk = "asUnOvShoot"] computes optimally robust influence curve for 
robust models with infinitesimal neighborhoods and
asymptotic under-/overshoot risk. 

\item[model = "FixRobModel", risk = "fiUnOvShoot"] computes optimally robust influence curve for 
robust models with fixed neighborhoods and
finite-sample under-/overshoot risk. 
}
\end{Section}
\begin{Author}\relax
Matthias Kohl \email{Matthias.Kohl@stamats.de}
\end{Author}
\begin{References}\relax
Huber, P.J. (1968) Robust Confidence Limits. Z. Wahrscheinlichkeitstheor.
Verw. Geb. \bold{10}:269--278.

Rieder, H. (1980) Estimates derived from robust tests. Ann. Stats. \bold{8}: 106--115.

Rieder, H. (1994) \emph{Robust Asymptotic Statistics}. New York: Springer.

Kohl, M. (2005) \emph{Numerical Contributions to the Asymptotic Theory of Robustness}. 
Bayreuth: Dissertation.
\end{References}
\begin{SeeAlso}\relax
\code{\LinkA{InfluenceCurve-class}{InfluenceCurve.Rdash.class}}, \code{\LinkA{RiskType-class}{RiskType.Rdash.class}}
\end{SeeAlso}
\begin{Examples}
\begin{ExampleCode}
B <- BinomFamily(size = 25, prob = 0.25) 

## classical optimal IC
IC0 <- optIC(model = B, risk = asCov())
plot(IC0) # plot IC
checkIC(IC0, B)
\end{ExampleCode}
\end{Examples}

