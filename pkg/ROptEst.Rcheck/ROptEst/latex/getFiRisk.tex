\HeaderA{getFiRisk}{Generic Function for Computation of Finite-Sample Risks}{getFiRisk}
\aliasA{getFiRisk,fiUnOvShoot,Norm,ContNeighborhood-method}{getFiRisk}{getFiRisk,fiUnOvShoot,Norm,ContNeighborhood.Rdash.method}
\aliasA{getFiRisk,fiUnOvShoot,Norm,TotalVarNeighborhood-method}{getFiRisk}{getFiRisk,fiUnOvShoot,Norm,TotalVarNeighborhood.Rdash.method}
\aliasA{getFiRisk-methods}{getFiRisk}{getFiRisk.Rdash.methods}
\begin{Description}\relax
Generic function for the computation of finite-sample risks.
This function is rarely called directly. It is used by 
other functions.
\end{Description}
\begin{Usage}
\begin{verbatim}
getFiRisk(risk, Distr, neighbor, ...)

## S4 method for signature 'fiUnOvShoot, Norm,
##   ContNeighborhood':
getFiRisk(risk, Distr, 
          neighbor, clip, stand, sampleSize, Algo, cont)

## S4 method for signature 'fiUnOvShoot, Norm,
##   TotalVarNeighborhood':
getFiRisk(risk, Distr, 
          neighbor, clip, stand, sampleSize, Algo, cont)
\end{verbatim}
\end{Usage}
\begin{Arguments}
\begin{ldescription}
\item[\code{risk}] object of class \code{"RiskType"}. 
\item[\code{Distr}] object of class \code{"Distribution"}. 
\item[\code{neighbor}] object of class \code{"Neighborhood"}. 
\item[\code{...}] additional parameters. 
\item[\code{clip}] positive real: clipping bound 
\item[\code{stand}] standardizing constant/matrix. 
\item[\code{sampleSize}] integer: sample size. 
\item[\code{Algo}] "A" or "B". 
\item[\code{cont}] "left" or "right". 
\end{ldescription}
\end{Arguments}
\begin{Details}\relax
The computation of the finite-sample under-/overshoot risk
is based on FFT. For more details we refer to Section 11.3 of Kohl (2005).
\end{Details}
\begin{Value}
The finite-sample risk is computed.
\end{Value}
\begin{Section}{Methods}
\describe{
\item[risk = "fiUnOvShoot", Distr = "Norm", neighbor = "ContNeighborhood"] computes finite-sample under-/overshoot risk in methods for 
function \code{getFixRobIC}. 

\item[risk = "fiUnOvShoot", Distr = "Norm", neighbor = "TotalVarNeighborhood"] computes finite-sample under-/overshoot risk in methods for 
function \code{getFixRobIC}. 
}
\end{Section}
\begin{Author}\relax
Matthias Kohl \email{Matthias.Kohl@stamats.de}
\end{Author}
\begin{References}\relax
Huber, P.J. (1968) Robust Confidence Limits. Z. Wahrscheinlichkeitstheor.
Verw. Geb. \bold{10}:269--278.

Kohl, M. (2005) \emph{Numerical Contributions to the Asymptotic Theory of Robustness}. 
Bayreuth: Dissertation.

Ruckdeschel, P. and Kohl, M. (2005) Computation of the Finite Sample Risk 
of M-estimators on Neighborhoods.
\end{References}
\begin{SeeAlso}\relax
\code{\LinkA{fiRisk-class}{fiRisk.Rdash.class}}
\end{SeeAlso}

