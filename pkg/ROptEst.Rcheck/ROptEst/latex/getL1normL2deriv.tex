\HeaderA{getL1normL2deriv}{Calculation of L1 norm of L2derivative}{getL1normL2deriv}
\aliasA{getL1normL2deriv,RealRandVariable-method}{getL1normL2deriv}{getL1normL2deriv,RealRandVariable.Rdash.method}
\aliasA{getL1normL2deriv,UnivariateDistribution-method}{getL1normL2deriv}{getL1normL2deriv,UnivariateDistribution.Rdash.method}
\aliasA{getL1normL2deriv-methods}{getL1normL2deriv}{getL1normL2deriv.Rdash.methods}
\begin{Description}\relax
Methods to calculate the L1 norm of the L2derivative in a smooth parametric model.
\end{Description}
\begin{Usage}
\begin{verbatim}getL1normL2deriv(L2deriv, ...)
## S4 method for signature 'UnivariateDistribution':
getL1normL2deriv(L2deriv, 
     cent, ...)

## S4 method for signature 'UnivariateDistribution':
getL1normL2deriv(L2deriv, 
     cent, stand, Distr, ...)

\end{verbatim}
\end{Usage}
\begin{Arguments}
\begin{ldescription}
\item[\code{L2deriv}] L2derivative of the model
\item[\code{cent}] centering Lagrange Multiplier
\item[\code{stand}] standardizing Lagrange Multiplier
\item[\code{Distr}] distribution of the L2derivative
\item[\code{...}] further arguments (not used at the moment)
\end{ldescription}
\end{Arguments}
\begin{Value}
L1 norm of the L2derivative
\end{Value}
\begin{Author}\relax
Peter Ruckdeschel \email{Peter.Ruckdeschel@uni-bayreuth.de}
\end{Author}
\begin{Examples}
\begin{ExampleCode}
##
\end{ExampleCode}
\end{Examples}

