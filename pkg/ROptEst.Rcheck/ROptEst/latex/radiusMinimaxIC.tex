\HeaderA{radiusMinimaxIC}{Generic function for the computation of the radius minimax IC}{radiusMinimaxIC}
\aliasA{radiusMinimaxIC,L2ParamFamily,UncondNeighborhood,asGRisk-method}{radiusMinimaxIC}{radiusMinimaxIC,L2ParamFamily,UncondNeighborhood,asGRisk.Rdash.method}
\aliasA{radiusMinimaxIC-methods}{radiusMinimaxIC}{radiusMinimaxIC.Rdash.methods}
\begin{Description}\relax
Generic function for the computation of the radius minimax IC.
\end{Description}
\begin{Usage}
\begin{verbatim}
radiusMinimaxIC(L2Fam, neighbor, risk, ...)

## S4 method for signature 'L2ParamFamily,
##   UncondNeighborhood, asGRisk':
radiusMinimaxIC(
        L2Fam, neighbor, risk, biastype = symmetricBias(),
        loRad, upRad, z.start = NULL, A.start = NULL, upper = 1e5, 
        maxiter = 100, tol = .Machine$double.eps^0.4, warn = FALSE)
\end{verbatim}
\end{Usage}
\begin{Arguments}
\begin{ldescription}
\item[\code{L2Fam}] L2-differentiable family of probability measures. 
\item[\code{neighbor}] object of class \code{"Neighborhood"}. 
\item[\code{risk}] object of class \code{"RiskType"}. 
\item[\code{...}] additional parameters. 
\item[\code{biastype}] object of class \code{"BiasType"}. 
\item[\code{loRad}] the lower end point of the interval to be searched. 
\item[\code{upRad}] the upper end point of the interval to be searched. 
\item[\code{z.start}] initial value for the centering constant. 
\item[\code{A.start}] initial value for the standardizing matrix. 
\item[\code{upper}] upper bound for the optimal clipping bound. 
\item[\code{maxiter}] the maximum number of iterations 
\item[\code{tol}] the desired accuracy (convergence tolerance).
\item[\code{warn}] logical: print warnings. 
\end{ldescription}
\end{Arguments}
\begin{Value}
The radius minimax IC is computed.
\end{Value}
\begin{Section}{Methods}
\describe{
\item[L2Fam = "L2ParamFamily", neighbor = "UncondNeighborhood", risk = "asGRisk":] computation of the radius minimax IC for an L2 differentiable parametric family. 
}
\end{Section}
\begin{Author}\relax
Matthias Kohl \email{Matthias.Kohl@stamats.de},
Peter Ruckdeschel \email{Peter.Ruckdeschel@uni-bayreuth.de}
\end{Author}
\begin{References}\relax
Rieder, H., Kohl, M. and Ruckdeschel, P. (2001) The Costs of not Knowing
the Radius. Submitted. Appeared as discussion paper Nr. 81. 
SFB 373 (Quantification and Simulation of Economic Processes),
Humboldt University, Berlin; also available under
\url{www.uni-bayreuth.de/departments/math/org/mathe7/RIEDER/pubs/RR.pdf}

Kohl, M. (2005) \emph{Numerical Contributions to the Asymptotic Theory of Robustness}. 
Bayreuth: Dissertation.
\end{References}
\begin{SeeAlso}\relax
\code{\LinkA{radiusMinimaxIC}{radiusMinimaxIC}}
\end{SeeAlso}
\begin{Examples}
\begin{ExampleCode}
N <- NormLocationFamily(mean=0, sd=1) 
radiusMinimaxIC(L2Fam=N, neighbor=ContNeighborhood(), 
                risk=asMSE(), loRad=0.1, upRad=0.5)
\end{ExampleCode}
\end{Examples}

