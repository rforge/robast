\HeaderA{getFixRobIC}{Generic Function for the Computation of Optimally Robust ICs}{getFixRobIC}
\aliasA{getFixRobIC,Norm,fiUnOvShoot,UncondNeighborhood-method}{getFixRobIC}{getFixRobIC,Norm,fiUnOvShoot,UncondNeighborhood.Rdash.method}
\aliasA{getFixRobIC-methods}{getFixRobIC}{getFixRobIC.Rdash.methods}
\begin{Description}\relax
Generic function for the computation of optimally robust ICs 
in case of robust models with fixed neighborhoods. This function is 
rarely called directly.
\end{Description}
\begin{Usage}
\begin{verbatim}
getFixRobIC(Distr, risk, neighbor, ...)

## S4 method for signature 'Norm, fiUnOvShoot,
##   UncondNeighborhood':
getFixRobIC(Distr, risk, neighbor, 
          sampleSize, upper, maxiter, tol, warn, Algo, cont)
\end{verbatim}
\end{Usage}
\begin{Arguments}
\begin{ldescription}
\item[\code{Distr}] object of class \code{"Distribution"}. 
\item[\code{risk}] object of class \code{"RiskType"}. 
\item[\code{neighbor}] object of class \code{"Neighborhood"}. 
\item[\code{...}] additional parameters. 
\item[\code{sampleSize}] integer: sample size. 
\item[\code{upper}] upper bound for the optimal clipping bound. 
\item[\code{maxiter}] the maximum number of iterations. 
\item[\code{tol}] the desired accuracy (convergence tolerance).
\item[\code{warn}] logical: print warnings. 
\item[\code{Algo}] "A" or "B". 
\item[\code{cont}] "left" or "right". 
\end{ldescription}
\end{Arguments}
\begin{Value}
The optimally robust IC is computed.
\end{Value}
\begin{Section}{Methods}
\describe{
\item[Distr = "Norm", risk = "fiUnOvShoot", neighbor = "UncondNeighborhood"] computes the optimally robust influence curve for one-dimensional
normal location and finite-sample under-/overshoot risk. 
}
\end{Section}
\begin{Author}\relax
Matthias Kohl \email{Matthias.Kohl@stamats.de}
\end{Author}
\begin{References}\relax
Huber, P.J. (1968) Robust Confidence Limits. Z. Wahrscheinlichkeitstheor.
Verw. Geb. \bold{10}:269--278.

Kohl, M. (2005) \emph{Numerical Contributions to the Asymptotic Theory of Robustness}. 
Bayreuth: Dissertation.
\end{References}
\begin{SeeAlso}\relax
\code{\LinkA{FixRobModel-class}{FixRobModel.Rdash.class}}
\end{SeeAlso}

