\HeaderA{lowerCaseRadius}{Computation of the lower case radius}{lowerCaseRadius}
\aliasA{lowerCaseRadius,L2ParamFamily,ContNeighborhood,asMSE,BiasType-method}{lowerCaseRadius}{lowerCaseRadius,L2ParamFamily,ContNeighborhood,asMSE,BiasType.Rdash.method}
\aliasA{lowerCaseRadius,L2ParamFamily,TotalVarNeighborhood,asMSE,BiasType-method}{lowerCaseRadius}{lowerCaseRadius,L2ParamFamily,TotalVarNeighborhood,asMSE,BiasType.Rdash.method}
\aliasA{lowerCaseRadius-methods}{lowerCaseRadius}{lowerCaseRadius.Rdash.methods}
\begin{Description}\relax
The lower case radius is computed; confer Subsection 2.1.2 
in Kohl (2005) and formula (4.5) in Ruckdeschel (2005).
\end{Description}
\begin{Usage}
\begin{verbatim}
lowerCaseRadius(L2Fam, neighbor, risk, biastype, ...)
\end{verbatim}
\end{Usage}
\begin{Arguments}
\begin{ldescription}
\item[\code{L2Fam}] L2 differentiable parametric family 
\item[\code{neighbor}] object of class \code{"Neighborhood"} 
\item[\code{risk}] object of class \code{"RiskType"} 
\item[\code{biastype}] object of class \code{"BiasType"} 
\item[\code{...}] additional parameters 
\end{ldescription}
\end{Arguments}
\begin{Value}
lower case radius
\end{Value}
\begin{Section}{Methods}
\describe{
\item[L2Fam = "L2ParamFamily", neighbor = "ContNeighborhood", risk = "asMSE",
biastype = "BiasType"] lower case radius for risk \code{"asMSE"} in case of \code{"ContNeighborhood"}
for symmetric bias.

\item[L2Fam = "L2ParamFamily", neighbor = "TotalVarNeighborhood", risk = "asMSE",
biastype = "BiasType"] lower case radius for risk \code{"asMSE"} in case of \code{"TotalVarNeighborhood"};
(argument biastype is just for signature reasons).
}
\end{Section}
\begin{Author}\relax
Matthias Kohl \email{Matthias.Kohl@stamats.de},
Peter Ruckdeschel \email{Peter.Ruckdeschel@uni-bayreuth.de}
\end{Author}
\begin{References}\relax
Kohl, M. (2005) \emph{Numerical Contributions to the Asymptotic Theory of Robustness}. 
Bayreuth: Dissertation.

Ruckdeschel, P. (2005) Optimally One-Sided Bounded Influence Curves.
Mathematical Methods in Statistics \emph{14}(1), 105-131.
\end{References}
\begin{SeeAlso}\relax
\code{\LinkA{L2ParamFamily-class}{L2ParamFamily.Rdash.class}}, \code{\LinkA{Neighborhood-class}{Neighborhood.Rdash.class}}
\end{SeeAlso}
\begin{Examples}
\begin{ExampleCode}
lowerCaseRadius(BinomFamily(size = 10), ContNeighborhood(), asMSE())
lowerCaseRadius(BinomFamily(size = 10), TotalVarNeighborhood(), asMSE())
\end{ExampleCode}
\end{Examples}

